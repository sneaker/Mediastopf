% Install the following packages: beamer, xcolor and pgm

\documentclass[]{beamer}
\mode<presentation>{
 \usetheme{Rochester}			% schlichtes, blaues theme
 \setbeamercovered{transparent}		% Schattieren des verborgenen Textes
}
\usepackage[T1]{fontenc}
\usepackage[utf8]{inputenc}		% Support german umlauts
\usepackage[ngerman]{babel}		% Print german date
%\usepackage{pgfpages}			% Ausdruck auf mehrere Seiten
%\pgfpagesuselayout{4 on 1}[a4paper,landscape,border shrink=5mm]
%\usepackage{graphicx}			% Graphiken benutzen

\title{MediaStopf -- Review Milestone 3}
\author{David Tran, Marco Birchler, \\ Thomas Kallenberg, Martin Schwab}
\institute{HSR Hochschule Rapperswil}
\date{ \today }

\begin{document}

\begin{frame}
  \titlepage
\end{frame}

\begin{frame}{Agenda}

\begin{itemize}[<+->]
\item Prototyp -- Demonstration
\item Ein interessanter Aspekt \\ (z.B. wie testet man Netzwerkverbindung)
\item Nächste Schritte
\end{itemize}

\end{frame}
\begin{frame}{Zweiter Titel}
ein bisserl text
\uncover<2->{Ab zweitem Klick anzeigen}
\invisible<1>{das ist versteckt} 
das ist sehr \alert{<2->wichtig}
\begin{theorem}
Das ist besonders interessant
\end{theorem}

\begin{enumerate}
\item erstens
\item zweitens
\item drittens
\end{enumerate}

% \minipage \includegraphics 10cm
% \includegraphics[scale=0.3]{bild}

\end{frame}

\end{document}

