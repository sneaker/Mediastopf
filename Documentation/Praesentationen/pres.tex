% Install the following packages: beamer, xcolor and pgm

\documentclass[]{beamer}
\mode<presentation>{
 \usetheme{Rochester}			% schlichtes, blaues theme
 \setbeamercovered{transparent}		% Schattieren des verborgenen Textes
}
\usepackage[T1]{fontenc}
\usepackage[utf8]{inputenc}		% Support german umlauts
\usepackage[ngerman]{babel}		% Print german date
%\usepackage{pgfpages}			% Ausdruck auf mehrere Seiten
%\pgfpagesuselayout{4 on 1}[a4paper,landscape,border shrink=5mm]
\usepackage{graphicx}			% Graphiken benutzen

\pgfdeclareimage[height=0.5cm]{logo}{logo}	% Logo rechts unten anzeigen
\logo{\pgfuseimage{logo}}

\title{MediaStopf -- Review Milestone 3}
\author{David Tran, Marco Birchler, \\ Thomas Kallenberg, Martin Schwab}
\institute{HSR Hochschule Rapperswil}
\date{ \today }

\begin{document}

\begin{frame}
  \titlepage
\end{frame}

\begin{frame}{Agenda}
%\framesubtitle{bla untertitel}
\begin{itemize} % [<+->]
%\item Architekturüberblick
\item ... was bisher geschah
\item Herausforderungen
\item Verbindung Client/Server
\item MediaStopf in Aktion
\item Nächste Schritte ...
\end{itemize}
\end{frame}

%\begin{frame}{Architekturüberblick}
%  \begin{figure}
%    \includegraphics[scale=0.2]{architektur}
%    \caption{Client- und Server-Architektur}
%  \end{figure}
%\end{frame}

\begin{frame}{... was bisher geschah}
\begin{columns}[t]
  \begin{column}{3cm}
    Wochen 1+2 \\
    Wochen 2+3 \\
    Wochen 4+5 \\
    Wochen 6+7 \\
  \end{column}
  \begin{column}{7cm}
    Idee, Domain-Modell \\
    Ausarbeitung Projektplan \\
    Entwicklung Prototypen \small{(Risiken)} \\
    1. Lauffähiges Gesamtsystem
  \end{column}
\end{columns}
\end{frame}

\begin{frame}{Herausforderungen}
\begin{itemize}
\item<1-> Eclipse und Git
\item<2-> Drittprogramme überwachen
\item<3-4> Huhn vs. Ei -- wer beginnt? \\ \small{(aka Schnittstellen)}
\invisible<1-4>{ \item . }
\invisible<1-3>{ \item und ...}
\end{itemize}
\end{frame}

\begin{frame}{Verbindung Client-/Server}
\begin{columns}[t]
\begin{column}{3cm}
  \begin{itemize}
   \item zwei Mal dasselbe?
   \item Symmetrie im Domainmodell...
   % TODO
   \item DM von ganz am Anfang wo wir gestritten haben über Import/Export
  \end{itemize}
\end{column}
\begin{column}{7cm}
  \begin{figure}
    \includegraphics[scale=0.26]{client_server}
    \caption{ähnlich? Client links, Server rechts}
  \end{figure}
\end{column}
\end{columns}
%\uncover<2->{Ab zweitem Klick anzeigen}
%\invisible<1>{das ist versteckt} 
\end{frame}

\begin{frame}{MediaStopf in Aktion}
% Max 3 Minuten probieren bis es geht, sonst mit den Slides demonstrieren!
\framesubtitle{Livedemo}
\end{frame}

\begin{frame}{MediaStopf in Aktion}
\framesubtitle{Offline-Variante -- Client}
\begin{columns}[t]
  \begin{column}{5cm}
    \begin{itemize}
      \item Eine CD wird eingelesen
    \end{itemize}
  \end{column}
  \begin{column}{5cm}
    \begin{figure}
      \includegraphics[scale=0.3]{client_tasks}
      \caption{Client beim Import}
    \end{figure}
  \end{column}
\end{columns}
\end{frame}

\begin{frame}{MediaStopf in Aktion}
\framesubtitle{Offline-Variante -- Server}
\begin{columns}[t]
  \begin{column}{5cm}
    \begin{itemize}
      \item Anbindung an Auftragsdatenbank
      \item Einsammeln der Dateien
    \end{itemize}
  \end{column}
  \begin{column}{5cm}
    \begin{figure}
      \includegraphics[scale=0.3]{server_2}
      \caption{Server mit Aufträgen}
    \end{figure}
  \end{column}
\end{columns}
\end{frame}

\begin{frame}{Und so gehts weiter...}
\begin{itemize}
 \item Volle Unterstützung der UC
 \item Systematisches Logging für Feedback bei Systemtest
 \item Refactoring, Subsysteme separiert halten
 \item Umstellung SQLite -> echte Datenbank
\end{itemize}
\end{frame}

\end{document}
